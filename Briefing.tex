\documentclass[12pt]{book}
\usepackage[T1]{fontenc}
\usepackage[utf8]{inputenc}
\usepackage[spanish]{babel}
\usepackage{hyperref}
\usepackage{vmargin}

\usepackage{titlesec}
\usepackage{listings}

\titleformat{\chapter}[display]
  {\normalfont\bfseries}{}{0pt}{\Huge}

\title {Práctica 2}
\date{21 de marzo}
\author{Francisco de Asís Carrasco Conde}

\begin{document}

\maketitle
\tableofcontents

\chapter{Creación de la calculadora}
    \section{Introducción}
        Para poder crear la calculadora vamos a tener que necesitar el archivo formato rpc o ".x".
        En este vamos a poner todas las estructuras de datos con las que vamos a trabajar, las funciones
        que va a realizar el servidor con sus numeros de procedimiento, las distintas versiones de la calculadora
        que vayamos a implementar y el numero de programa.

        En mi caso voy a usar dos estructuras:
    \begin{lstlisting}

    struct result{
        MiVector result_vect;
        double result_value;
        int code;
    };

    struct MiVector{
        double a;
        double b; 
        double c;
    };
    \end{lstlisting}

        También voy a implementar 2 versiones de la calculadora, una para los cálculos simples y otra para cálculos
        con vectores. En el archivo rpc quedaría tal que así:

        \newpage
        \begin{lstlisting}

program CALCULADORA {
    version CALCULADORA_1 {
        result suma(double val1, double val2) = 1;
        result restar(double val1, double val2) = 2;
        result multiplica(double val1, double val2) = 3;
        result divide(double val1, double val2) = 4;
    } = 1;
        
    version CALCULADORA_2{
        result suma(MiVector v1, MiVector v2) = 1;
        result restar(MiVector v1, MiVector v2) = 2;
        result multiplica(MiVector v1, MiVector v2) = 3;
    } = 2;
        
} = 0x20000001;

        \end{lstlisting}

Después de haber creado el archivo ejecutamos el comando:
\begin{lstlisting}
# rcpgen -NCa calculadora.x
\end{lstlisting}
Y se nos crearan todos los archivos que tendremos que ir rellenando con código c.

\section{Contenido servidor.c}
    En el servidor.c generado por rpc se nos creará una plantilla con las funciones a
    rellenar, en estas ponemos el codigo con lo que queramos que haga cada función, por
    ejemplo en suma\_1\_svc, el código para sumar dos numeros. En suma\_2\_svc, tengo el código
    para sumar 2 vectores. Y asi respectivamente.

\section{Contenido cliente.c}
    En cliente.c vamos a rellenar dos cosas, el main y las funciones de las distintas versiones,
    en mi caso son 2. La calculadora\_1 es para las operaciones básicas y la 2 para operaciones
    con vectores.

    Lo que hay que rellenar en las versiones de las calculadoras son llamadas a las funciones del servidor, La
    evaluacion de respuesta por si devuelve algún código de error y el retorno del valor deseado.

    En el main la usaremos para filtrar la entrada de datos del usuario. Si se pone un número de argumentos y/o argumentos
    incorrectos devolverá un error.

\chapter{Forma de usar la calculadora}

\section{Calculadora Básica}

Los números que se pueden meten por argumentos son tipo double y podemos sumar, restar, multiplicar y dividir.
El formato es el siguiente:

\begin{lstlisting}
# ./calculadora_client <servidor> <numero1> <signoOperacion> <numero2>
\end{lstlisting}

\section{Calculadora Vectores}

Para acceder a esta versión de la calculadora hay que poner lo siguiente:

\begin{lstlisting}
# ./calculadora_client <servidor> V
\end{lstlisting}

Una vez hayamos introducido el comando se nos preguntará con que números queremos operar. El tamaño
de los vectores es fijo: 3 componentes. Los numeros se meten uno seguido de otro.

Luego nos pedirá la operación que queramos realizar, en esta versión hay se puede sumar, restar y multiplicar.
La multiplicación es vectorial.

\end{document}